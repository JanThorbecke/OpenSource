\newpage
\section{onewvsp} \label{onewvsp}

\subsection{General}

{\bf onewvsp} is based on the same algorithm as used in \htmlref{{\bf extrap}}{extrap}. {\bf onewvsp} can be used to generate pseudo VSP
data from seismic surface data. For the extrapolation of the data
one-way extrapolation operators (these operators can be calculated
with \htmlref{{\bf opercalc}}{opercalc}) are used.

\subsection{Parameters}

Via the command-line or in a parameter file: {\tt par=$<$parameter\_file$>$}.
{\footnotesize
\begin{verbatim}
 onewvsp - One-way VSP generation
 
 onewvsp file_in= file_vel= [optional parameters]
 
 Required parameters:
 
   file_in= ................. Input file
   file_vel= ................ gridded velocity file 
 
 Optional parameters:
 
   file_vsp= ................ Output file of calculated VSP
   file_ex= ................. Output file with the extrapolated result
   file_over= ............... writes model file with vsp positions
   nxmax=512 ................ maximum number of traces in input file
   ntmax=1024 ............... maximum number of samples/trace in input file
   file_init= ............... filename for ProMax IO initialization
   line=1  .................. 1: black lines; 0: white lines in overlay
   verbose=0 ................ silent option; >0 display info
 RECEIVER POSITIONS 
   xrcv=0 ................... x-position's of receivers (array)
   zrcv=0,nz*dz ............. z-position of the receivers (array)
   dxrcv=dx ................. step in receiver x-direction
   dzrcv=dz ................. step in receiver z-direction
   lint=1 ................... linear interpolate between the rcv points
   dxspr=0 .................. step of receiver spread in x-direction
   nvsp=1 ................... number of VSP positions
 EXTRAPOLATION 
   mode=-1 .................. type of extrapolation (1=forward, -1=inverse)
   fmin=0 ................... minimum frequency
   fmax=70 .................. maximum frequency
   ntap=0 ................... number of taper points at boundaries
 EXTRAPOLATION OPERATOR DEFINITION 
   select=4 ................. type of x-w operator
   opl=25 ................... length of the convolution operator (odd)
   alpha=65 ................. maximum angle of interest
   perc=0.15 ................ smoothness of filter edge
   weight=5e-5 .............. weight factor in WLSQ operator calculation
   fine=10 .................. fine sampling in operator table
   filter=1 ................. apply kx-w filter to desired operator
   limit=1.0002.............. maximum amplitude in best operators
   opl_min=15 ............... minimum length of convolution operator
  
   Options for select:
         - 0 = Truncated operator
         - 1 = Gaussian tapered operator
         - 2 = Kaiser tapered operator
         - 3 = Smoothed Phase operator
         - 4 = Weighted Least Squares operator
         - 5 = Remez exchange operator
         - 8 = Smooth Weighted Least Squares operator
         - 9 = Optimum Smooth Weighted Least Squares operator
         - 10= Optimum Weighted Least Squares operator
 
   The weighting factor is used in the convolution operator calculation.
   This calculation is done in an optimized way.
   The default weight factor is for most cases correct. For a more stable
   operator chooce a weight factor closer to 1, if 1 is chosen no 
   optimization is carried out and the convolution operator is the 
   truncated Inverse Fourier Transform of the Kx-w operator.
   The non-optimized operator is the truncated IFFT of a smooth Kx-w 
   operator. This operator is designed by Gerrit Blacquiere.
 
   Note that all coordinates are related to the velocity model.
 
  Copyright 1997, 2008 Jan Thorbecke, (janth@xs4all.nl) 
 
      initial version   : 14-12-1993 (j.w.thorbecke@tudelft.nl)
          version 1.0   : 11-10-1995 (release version)
          version 2.0   : 23-06-2008 (janth@xs4all.nl) 2008
\end{verbatim}}

\noindent

\subsection{General parameter description}

The input files are the same as for the program \htmlref{{\bf
    extrap}}{extrap}. The data to be extrapolated ({\tt file\_in}) and
the gridded subsurface ({\tt file\_vel}) should at least have the same
number of traces. If the number of traces of the subsurface is smaller
an error message is the result. If the number of traces is bigger then
the user should position the first receiver of the data in the
subsurface grid. In no correct hdrs ({\tt gx} and {\tt sx}) are
available this can be done with the parameter {\tt rpos}, which
indicates the tracenumber (an integer) in the subsurface grid where
the first receiver is positioned. The distance between the traces in
both gathers should be equal. The extrapolation direction is
controlled with {\tt mode=}. For inverse extrapolation the complex
conjugate of the forward extrapolation operator is taken.

To avoid reflections at the edges of the model the parameter {\tt
  ntap} can be set. {\tt ntap} indicates the number of points at the
edges for which a spatial taper is designed according to:
$\exp{(-(0.4*(ntap-ix)/ntap)^2)}$. Choosing {\tt ntap} equal to half
of the operator length is an optimimum value.

The output file {\tt file\_vsp} contains the pseudo VSP records for
the different VSP positions. If {\tt file\_over} is defined the VSP
positions are overlayed on the gridded velocity model. Displaying the
{\tt file\_over} file gives an overview of the chosen VSP positions.
{\tt file\_ex} (if defined) contains the extrapolated data which is
extrapolated upto the deepest receiver in the VSP array.

The receiver positions of the VSP array are defined with the
parameters {\tt xrcv}, {\tt zrcv}, {\tt dxrcv}, {\tt dzrcv}, and {\tt
  lint}. If {\tt xrcv} and {\tt zrcv} are not defined the default
receiver array is calculated. This receiver array is positioned at x=0
for all depth positions in the gridded subsurface model. If {\tt xrcv}
is defined with only one value (e.g. {\tt xrcv=1500}) then the VSP is
positioned at x=xrcv for all defined depth positions. Note that {\tt
  dzrcv} defines at which depth postions receivers should be placed.
Choosing an array for {\tt zrcv} (e.g. {\tt zrcv=0,2000}) gives only
positions inbetween the defined depth array. For a deviated VSP
configuration both {\tt xrcv} and {\tt zrcv} should be defined as
arrays (e.g. {\tt xrcv=1000,1000,1500 zrcv=0,2000,3000}). The
parameter {\tt lint} set to 1 calculates (with {\tt dxrcv} and {\tt
  dzrcv} defined) inbetween the given positions the receiver array
(use the {\tt file\_over} option to see how the receivers are
positioned). If {\tt lint} is set to 0 then only the receiver postions
at every pair of {\tt xrcv} and {\tt zrcv} are chosen (in the previous
example only three postions).

The parameters {\tt nvsp} and {\tt dxspr} give the posibility to
define more than one receiver arrays, {\tt nvsp} defines the number of
arrays and {\tt dxspr} gives the distance between the receiver arrays.
The first receiver array is defined with {\tt xrcv}, {\tt zrcv}, {\tt
  dxrcv}, {\tt dzrcv}, and {\tt lint} (see above). The other receiver
arrays have the same structure but are calculated at x-positions which
are dxspr moved (use the \\ {\tt file\_over} option to see how the
receivers are positioned).

For a more detailed discussion on the different parameters which are
related to the operator optimization the reader is referred to the
description of the program \htmlref{{\bf opercalc}}{opercalc}.

\subsection{Examples}

{\footnotesize
\begin{verbatim}onewvsp file_vel= file_vsp= file_over= 
\end{verbatim}}

\subsection{To do}
Nothing really.

